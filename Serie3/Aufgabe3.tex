\documentclass{article}

\usepackage{amsmath}
\usepackage{amsfonts}
\usepackage{amssymb}

\begin{document}

\section{3a)}

Sei $K: Graph \to \mathbb{N}_{0}$ die Anzahl der Kreise.

Sein $S$, $S^{\prime}$  Mengen von Kanten.

Sei $G-S$ der Graph $G$ ohne die Kanten in $S$.

Sei $G+S$ der Graph $G$ um die Kanten in $S$ erweitert.

Dann ist $K(G) \geq K(G-S)$ und $K(G) \leq K(G+S)$.

Dann entspricht das Anwenden eines Feedback Sets $F$ dem entfernen von einer Kanten Menge $S$
und dem hinzufügen einer Kanten Menge $S^{\prime}$.

Wobei $S^{\prime} = \{(e,f)|(f,e) \in S\}$.

Dann gilt:

$K(G) \geq K(G-S) \leq K(G-S+S^{\prime}) = 0$
$\Rightarrow K(G-S) = 0$

$K(G-S) = 0$ ist Kreiszahl nach der Anwendung von $F$ als Feedback Arc Set, somit F auch Feedback Arc Set.

\section{3b)}

Sei G gerichter Graph.

Sei FS minimal Feedback Set.

\subsection{Behauptung}

FS ist minimales Feedback Arc Set.

\subsection{Beweis}



\end{document}